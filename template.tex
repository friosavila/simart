% Options for packages loaded elsewhere
\PassOptionsToPackage{unicode}{hyperref}
\PassOptionsToPackage{hyphens}{url}
\PassOptionsToPackage{dvipsnames,svgnames,x11names}{xcolor}
%
\documentclass[
  11pts,
]{article}

\usepackage{amsmath,amssymb}
\usepackage{iftex}
\ifPDFTeX
  \usepackage[T1]{fontenc}
  \usepackage[utf8]{inputenc}
  \usepackage{textcomp} % provide euro and other symbols
\else % if luatex or xetex
  \usepackage{unicode-math}
  \defaultfontfeatures{Scale=MatchLowercase}
  \defaultfontfeatures[\rmfamily]{Ligatures=TeX,Scale=1}
\fi
\usepackage{lmodern}
\ifPDFTeX\else  
    % xetex/luatex font selection
\fi
% Use upquote if available, for straight quotes in verbatim environments
\IfFileExists{upquote.sty}{\usepackage{upquote}}{}
\IfFileExists{microtype.sty}{% use microtype if available
  \usepackage[]{microtype}
  \UseMicrotypeSet[protrusion]{basicmath} % disable protrusion for tt fonts
}{}
\makeatletter
\@ifundefined{KOMAClassName}{% if non-KOMA class
  \IfFileExists{parskip.sty}{%
    \usepackage{parskip}
  }{% else
    \setlength{\parindent}{0pt}
    \setlength{\parskip}{6pt plus 2pt minus 1pt}}
}{% if KOMA class
  \KOMAoptions{parskip=half}}
\makeatother
\usepackage{xcolor}
\usepackage[lmargin=1in,rmargin=1in,tmargin=1in,bmargin=1in]{geometry}
\setlength{\emergencystretch}{3em} % prevent overfull lines
\setcounter{secnumdepth}{5}
% Make \paragraph and \subparagraph free-standing
\ifx\paragraph\undefined\else
  \let\oldparagraph\paragraph
  \renewcommand{\paragraph}[1]{\oldparagraph{#1}\mbox{}}
\fi
\ifx\subparagraph\undefined\else
  \let\oldsubparagraph\subparagraph
  \renewcommand{\subparagraph}[1]{\oldsubparagraph{#1}\mbox{}}
\fi

\usepackage{color}
\usepackage{fancyvrb}
\newcommand{\VerbBar}{|}
\newcommand{\VERB}{\Verb[commandchars=\\\{\}]}
\DefineVerbatimEnvironment{Highlighting}{Verbatim}{commandchars=\\\{\}}
% Add ',fontsize=\small' for more characters per line
\usepackage{framed}
\definecolor{shadecolor}{RGB}{241,243,245}
\newenvironment{Shaded}{\begin{snugshade}}{\end{snugshade}}
\newcommand{\AlertTok}[1]{\textcolor[rgb]{0.68,0.00,0.00}{#1}}
\newcommand{\AnnotationTok}[1]{\textcolor[rgb]{0.37,0.37,0.37}{#1}}
\newcommand{\AttributeTok}[1]{\textcolor[rgb]{0.40,0.45,0.13}{#1}}
\newcommand{\BaseNTok}[1]{\textcolor[rgb]{0.68,0.00,0.00}{#1}}
\newcommand{\BuiltInTok}[1]{\textcolor[rgb]{0.00,0.23,0.31}{#1}}
\newcommand{\CharTok}[1]{\textcolor[rgb]{0.13,0.47,0.30}{#1}}
\newcommand{\CommentTok}[1]{\textcolor[rgb]{0.37,0.37,0.37}{#1}}
\newcommand{\CommentVarTok}[1]{\textcolor[rgb]{0.37,0.37,0.37}{\textit{#1}}}
\newcommand{\ConstantTok}[1]{\textcolor[rgb]{0.56,0.35,0.01}{#1}}
\newcommand{\ControlFlowTok}[1]{\textcolor[rgb]{0.00,0.23,0.31}{#1}}
\newcommand{\DataTypeTok}[1]{\textcolor[rgb]{0.68,0.00,0.00}{#1}}
\newcommand{\DecValTok}[1]{\textcolor[rgb]{0.68,0.00,0.00}{#1}}
\newcommand{\DocumentationTok}[1]{\textcolor[rgb]{0.37,0.37,0.37}{\textit{#1}}}
\newcommand{\ErrorTok}[1]{\textcolor[rgb]{0.68,0.00,0.00}{#1}}
\newcommand{\ExtensionTok}[1]{\textcolor[rgb]{0.00,0.23,0.31}{#1}}
\newcommand{\FloatTok}[1]{\textcolor[rgb]{0.68,0.00,0.00}{#1}}
\newcommand{\FunctionTok}[1]{\textcolor[rgb]{0.28,0.35,0.67}{#1}}
\newcommand{\ImportTok}[1]{\textcolor[rgb]{0.00,0.46,0.62}{#1}}
\newcommand{\InformationTok}[1]{\textcolor[rgb]{0.37,0.37,0.37}{#1}}
\newcommand{\KeywordTok}[1]{\textcolor[rgb]{0.00,0.23,0.31}{#1}}
\newcommand{\NormalTok}[1]{\textcolor[rgb]{0.00,0.23,0.31}{#1}}
\newcommand{\OperatorTok}[1]{\textcolor[rgb]{0.37,0.37,0.37}{#1}}
\newcommand{\OtherTok}[1]{\textcolor[rgb]{0.00,0.23,0.31}{#1}}
\newcommand{\PreprocessorTok}[1]{\textcolor[rgb]{0.68,0.00,0.00}{#1}}
\newcommand{\RegionMarkerTok}[1]{\textcolor[rgb]{0.00,0.23,0.31}{#1}}
\newcommand{\SpecialCharTok}[1]{\textcolor[rgb]{0.37,0.37,0.37}{#1}}
\newcommand{\SpecialStringTok}[1]{\textcolor[rgb]{0.13,0.47,0.30}{#1}}
\newcommand{\StringTok}[1]{\textcolor[rgb]{0.13,0.47,0.30}{#1}}
\newcommand{\VariableTok}[1]{\textcolor[rgb]{0.07,0.07,0.07}{#1}}
\newcommand{\VerbatimStringTok}[1]{\textcolor[rgb]{0.13,0.47,0.30}{#1}}
\newcommand{\WarningTok}[1]{\textcolor[rgb]{0.37,0.37,0.37}{\textit{#1}}}

\providecommand{\tightlist}{%
  \setlength{\itemsep}{0pt}\setlength{\parskip}{0pt}}\usepackage{longtable,booktabs,array}
\usepackage{calc} % for calculating minipage widths
% Correct order of tables after \paragraph or \subparagraph
\usepackage{etoolbox}
\makeatletter
\patchcmd\longtable{\par}{\if@noskipsec\mbox{}\fi\par}{}{}
\makeatother
% Allow footnotes in longtable head/foot
\IfFileExists{footnotehyper.sty}{\usepackage{footnotehyper}}{\usepackage{footnote}}
\makesavenoteenv{longtable}
\usepackage{graphicx}
\makeatletter
\def\maxwidth{\ifdim\Gin@nat@width>\linewidth\linewidth\else\Gin@nat@width\fi}
\def\maxheight{\ifdim\Gin@nat@height>\textheight\textheight\else\Gin@nat@height\fi}
\makeatother
% Scale images if necessary, so that they will not overflow the page
% margins by default, and it is still possible to overwrite the defaults
% using explicit options in \includegraphics[width, height, ...]{}
\setkeys{Gin}{width=\maxwidth,height=\maxheight,keepaspectratio}
% Set default figure placement to htbp
\makeatletter
\def\fps@figure{htbp}
\makeatother
% definitions for citeproc citations
\NewDocumentCommand\citeproctext{}{}
\NewDocumentCommand\citeproc{mm}{%
  \begingroup\def\citeproctext{#2}\cite{#1}\endgroup}
\makeatletter
 % allow citations to break across lines
 \let\@cite@ofmt\@firstofone
 % avoid brackets around text for \cite:
 \def\@biblabel#1{}
 \def\@cite#1#2{{#1\if@tempswa , #2\fi}}
\makeatother
\newlength{\cslhangindent}
\setlength{\cslhangindent}{1.5em}
\newlength{\csllabelwidth}
\setlength{\csllabelwidth}{3em}
\newenvironment{CSLReferences}[2] % #1 hanging-indent, #2 entry-spacing
 {\begin{list}{}{%
  \setlength{\itemindent}{0pt}
  \setlength{\leftmargin}{0pt}
  \setlength{\parsep}{0pt}
  % turn on hanging indent if param 1 is 1
  \ifodd #1
   \setlength{\leftmargin}{\cslhangindent}
   \setlength{\itemindent}{-1\cslhangindent}
  \fi
  % set entry spacing
  \setlength{\itemsep}{#2\baselineskip}}}
 {\end{list}}
\usepackage{calc}
\newcommand{\CSLBlock}[1]{\hfill\break\parbox[t]{\linewidth}{\strut\ignorespaces#1\strut}}
\newcommand{\CSLLeftMargin}[1]{\parbox[t]{\csllabelwidth}{\strut#1\strut}}
\newcommand{\CSLRightInline}[1]{\parbox[t]{\linewidth - \csllabelwidth}{\strut#1\strut}}
\newcommand{\CSLIndent}[1]{\hspace{\cslhangindent}#1}

\usepackage{lipsum}
\makeatletter
\@ifpackageloaded{caption}{}{\usepackage{caption}}
\AtBeginDocument{%
\ifdefined\contentsname
  \renewcommand*\contentsname{Table of contents}
\else
  \newcommand\contentsname{Table of contents}
\fi
\ifdefined\listfigurename
  \renewcommand*\listfigurename{List of Figures}
\else
  \newcommand\listfigurename{List of Figures}
\fi
\ifdefined\listtablename
  \renewcommand*\listtablename{List of Tables}
\else
  \newcommand\listtablename{List of Tables}
\fi
\ifdefined\figurename
  \renewcommand*\figurename{Figure}
\else
  \newcommand\figurename{Figure}
\fi
\ifdefined\tablename
  \renewcommand*\tablename{Table}
\else
  \newcommand\tablename{Table}
\fi
}
\@ifpackageloaded{float}{}{\usepackage{float}}
\floatstyle{ruled}
\@ifundefined{c@chapter}{\newfloat{codelisting}{h}{lop}}{\newfloat{codelisting}{h}{lop}[chapter]}
\floatname{codelisting}{Listing}
\newcommand*\listoflistings{\listof{codelisting}{List of Listings}}
\makeatother
\makeatletter
\makeatother
\makeatletter
\@ifpackageloaded{caption}{}{\usepackage{caption}}
\@ifpackageloaded{subcaption}{}{\usepackage{subcaption}}
\makeatother
\ifLuaTeX
  \usepackage{selnolig}  % disable illegal ligatures
\fi
\usepackage{bookmark}

\IfFileExists{xurl.sty}{\usepackage{xurl}}{} % add URL line breaks if available
\urlstyle{same} % disable monospaced font for URLs
\hypersetup{
  pdftitle={Simart Template},
  pdfauthor={Sarah Malloc; Eliza Dealloc},
  pdfkeywords={template, demo},
  colorlinks=true,
  linkcolor={blue},
  filecolor={Maroon},
  citecolor={Blue},
  urlcolor={Blue},
  pdfcreator={LaTeX via pandoc}}


\title{Simart Template}
\author{
Sarah Malloc\\
An Organization\\
Boston\\
\href{mailto:sm@example.org}{sm@example.org}\and 
Eliza Dealloc\\
Another Affiliation\\
\\
}
\date{}
\begin{document}


\def\spacingset#1{\renewcommand{\baselinestretch}%
{#1}\small\normalsize} \spacingset{1}

%Ipsum lorem

\maketitle
\begin{abstract}
This document is a template demonstrating the Simart format. In addition
to few tips on how to use quarto for producing articles.
\end{abstract}
 
\vspace{.2in}

\textbf{\textit{Keyword: }}
    template, 
    demo 


\thispagestyle{empty}
\clearpage\pagenumbering{arabic}
\newpage
\spacingset{1.2} % DON'T change the spacing!
\section{Introduction}\label{sec-intro}

This is a very simple template for the creation of academic articles. It
is based on the documentclass article. It has also been adapted so it
includes the information of authors, thank you note, corresponding
author information, etc. Using APA7th for citations.

It also provides a quick review of some of the options I learn to use to
create sections of interest. Just for fun, it also generates LIPSUM
text, to fill the rest of the text.

\section{Citing documents}\label{sec-cite}

There are two ways to cite documents. One is using standard author(year)
format, as well as a (author, year) format.

In the first case you use \texttt{@CameronTrivedi2013} to obtain Cameron
and Trivedi (2013).

In the second case you use \texttt{{[}@CameronTrivedi2013{]}} to obtain
(Cameron and Trivedi, 2013).

\section{Math}\label{math}

You can write math in-line using latex standard syntax. For example
\texttt{\$\textbackslash{}beta\$} would create \(\beta\).

To write longer equations you need to use \texttt{\$\$} before and after
the expression. I find it useful to start a new line to do so. As in the
following example:

\begin{Shaded}
\begin{Highlighting}[]
\SpecialStringTok{$$y = }\SpecialCharTok{\textbackslash{}beta}\SpecialStringTok{ X + }\SpecialCharTok{\textbackslash{}varepsilon}
\SpecialStringTok{$$}
\end{Highlighting}
\end{Shaded}

\[y = \beta X + \varepsilon
\]

For multiple equation formulas, it may be convinient to use
\texttt{\textbackslash{}begin\{aligned\}} and
\texttt{\textbackslash{}end\{aligned\}}.

\begin{Shaded}
\begin{Highlighting}[]
\SpecialStringTok{$$}
\KeywordTok{\textbackslash{}begin}\NormalTok{\{}\ExtensionTok{aligned}\NormalTok{\}}
\SpecialStringTok{y \&= }\SpecialCharTok{\textbackslash{}beta}\SpecialStringTok{ X + }\SpecialCharTok{\textbackslash{}varepsilon}\SpecialStringTok{ }\SpecialCharTok{\textbackslash{}\textbackslash{}}
\SpecialCharTok{\textbackslash{}varepsilon}\SpecialStringTok{ \&}\SpecialCharTok{\textbackslash{}sim}\SpecialStringTok{ N(0,}\SpecialCharTok{\textbackslash{}sigma}\SpecialStringTok{\^{}2)}
\KeywordTok{\textbackslash{}end}\NormalTok{\{}\ExtensionTok{aligned}\NormalTok{\}}
\SpecialStringTok{$$}
\end{Highlighting}
\end{Shaded}

\[
\begin{aligned}
y &= \beta X + \varepsilon \\
\varepsilon &\sim N(0,\sigma^2)
\end{aligned}
\]

These equations can be easily cross referenced using
\texttt{\{\#eq-something\}} after \texttt{\$\$}. Instead of something,
we just need a unique id to call to. For example:

\begin{Shaded}
\begin{Highlighting}[]
\SpecialStringTok{$$}
\KeywordTok{\textbackslash{}begin}\NormalTok{\{}\ExtensionTok{aligned}\NormalTok{\}}
\SpecialStringTok{y \&= }\SpecialCharTok{\textbackslash{}beta}\SpecialStringTok{ X + }\SpecialCharTok{\textbackslash{}varepsilon}\SpecialStringTok{ }\SpecialCharTok{\textbackslash{}\textbackslash{}}
\SpecialCharTok{\textbackslash{}varepsilon}\SpecialStringTok{ \&}\SpecialCharTok{\textbackslash{}sim}\SpecialStringTok{ N(0,}\SpecialCharTok{\textbackslash{}sigma}\SpecialStringTok{\^{}2)}
\KeywordTok{\textbackslash{}end}\NormalTok{\{}\ExtensionTok{aligned}\NormalTok{\}}
\SpecialStringTok{$$}\NormalTok{\{\#eq{-}ols\}}
\end{Highlighting}
\end{Shaded}

\begin{equation}\phantomsection\label{eq-ols}{
\begin{aligned}
y &= \beta X + \varepsilon \\
\varepsilon &\sim N(0,\sigma^2)
\end{aligned}
}\end{equation}

This can now be referenced using \texttt{@eq-ols}, this will provide
Equation~\ref{eq-ols}.

\section{Tables and cross-references}\label{tables-and-cross-references}

Starting with Quarto 1.4 it is really simple to add tables and
cross-reference in text.

\subsection{Adding pictures as tables}\label{adding-pictures-as-tables}

\begin{table}

\caption{\label{tbl-table1}}

\centering{

\captionsetup{labelsep=none}\includegraphics{pic_table.png}

}

\end{table}%

\begin{Shaded}
\begin{Highlighting}[]
\NormalTok{::: \{\#tbl{-}table1\}}
\NormalTok{![](pic\_table.png)\{fig{-}align=center\}}
\NormalTok{Table made with a pciture}
\NormalTok{:::}
\end{Highlighting}
\end{Shaded}

\texttt{@tbl-table1}: Table~\ref{tbl-table1} is a table that was
constructed in Excel, saved as figure, and reused in quarto.

\subsection{Adding Mark down as
tables}\label{adding-mark-down-as-tables}

\subsection{Adding HTML as tables}\label{adding-html-as-tables}

\subsection{Adding Latex as tables}\label{adding-latex-as-tables}

\section{Adding References section}\label{adding-references-section}

to add reference section at the end of your document, you can use the
following code:

\begin{Shaded}
\begin{Highlighting}[]
\FunctionTok{\# References \{.unnumbered\}}

\NormalTok{::: \{\#refs\}}
\NormalTok{:::}
\end{Highlighting}
\end{Shaded}

The option \texttt{\{.unnumbered\}} request not to add a number to the
section. whereas the fenced code request to add the references in the
section of your choice.

\section*{References}\label{references}
\addcontentsline{toc}{section}{References}

\phantomsection\label{refs}
\begin{CSLReferences}{1}{0}
\bibitem[\citeproctext]{ref-CameronTrivedi2013}
Cameron, A. C., and Trivedi, P. K. (2013). \emph{Regression analysis of
count data} (2nd ed.). Cambridge University Press.

\end{CSLReferences}

\newpage{}

\appendix

\section{Appendix on how to add an
appendix}\label{appendix-on-how-to-add-an-appendix}

I find that adding an appendix requires to pieces of code. First, after
reference one may add \texttt{} to start a new page. And use the latex
code \texttt{\textbackslash{}appendix}, to start the appendix section.

\section{second appendix}\label{second-appendix}

\subsection{sub appendix}\label{sub-appendix}



\end{document}
